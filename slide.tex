\documentclass[c,12pt]{beamer}
%\documentclass[hyperref={pdfpagemode=FullScreen,colorlinks=true},xcolor=table]{beamer}
\usepackage[utf8]{inputenc}
\usepackage[francais]{babel}
\usepackage[T1]{fontenc}
\usepackage{amsmath}
\usepackage{amsfonts}
\usepackage{amssymb}
\usepackage{beamerthemesplit}%pour avoir un joli plan sur tous les slides
%\usetheme{Copenhagen}
\usecolortheme{orchid}%un theme joli et une couleur
%\usefonttheme{serif}% pour les queues des lettres
%\usepackage{url}
%\let\urlorig\url
%\renewcommand{\url}[1]{
%  \begin{otherlanguage}{english}\urlorig{#1}\end{otherlanguage}%
%}
\usetheme{Warsaw}
\usecolortheme{seahorse}

%\addtobeamertemplate{footline}{\hfill \insertframenumber/\inserttotalframenumber}
%\setbeamertemplate{navigation symbols}{\insertframenumber}
\addtobeamertemplate{footline}{%
	 \hfill \insertframenumber/\inserttotalframenumber
	
    %\vskip1pt%
    %\setbeamertemplate{footline}[page number]%
    %\setbeamercolor{page number in head/foot}{use=headline,fg=headline.fg,bg=headline.bg}
    %\usebeamertemplate{footline}%    
}{% don't add anything after
}

\title[]{Belote (Machine Learning)}
\author[]{Morgane AUSTERN, Benjamin DONNOT, Loïc MICHEL}
\institute{ENSAE}

\begin{document}
\frame[plain]%pour que la première page prenne toute la place
{\titlepage}% pour faire une page de titre
\section{Introduction}
	\subsection{Plan}
\begin{frame}{Plan}
 \begin{itemize}
    \onslide<2-> {\item La structure du code}
    	\newline
    \onslide<3-> {\item L'interface graphique}
    	\newline
    \onslide<4-> {\item L'IA et le "machine learning"}
  \end{itemize}
\end{frame}

	\subsection{Mode de jeu}
\begin{frame}{Mode de jeu}
 \begin{itemize}
    \onslide<2-> {\item Belote "classique"}
    	\newline
    	
    \onslide<3-> {\item 4 joueurs IA ou 1 joueur humain et 3 joueurs humains}
    	\newline
    	
    \onslide<4-> {\item Pas de jeu en réseau, aucune possibilité de jouer à 2 humains}
    	\newline    	
  \end{itemize}
\end{frame}
	\subsection{TortoiseSVN}
\begin{frame}{TortoiseSVN}
 \begin{itemize}
    \onslide<2-> {\item Nous avons utilisé un outil de versionning}
    	\newline
    	
    \onslide<3-> {\item Plus d'efficacité}
    	\newline
    	
    \onslide<4-> {\item Moins d'envois de mails...}
    	\newline    	
  \end{itemize}
\end{frame}

\section{La structure du code}
\begin{frame}{\textbf{La structure du code} : Plan}
\begin{itemize}
	\onslide<2-> {\item La classe Carte }
	\newline
	\onslide<3-> {\item Le jeu }
	\newline
	\onslide<4-> {\item Les joueurs }
\end{itemize}
\end{frame}
	\subsection{Les cartes}
\begin{frame}{La classe Carte}
\begin{itemize}
	\onslide<2-> {\item Classe de base du jeu}
	\newline
	\onslide<3-> {\item Nécessité de matrices (quelle hauteur gagne)}
	\newline
	\onslide<4-> {\item Introduction de pseudos hauteurs}
\end{itemize}
\end{frame}
	
	\subsection{Le jeu}
\begin{frame}{Classe Jeu}
\begin{itemize}
	\onslide<2-> {\item Squelette du programme}
	\newline
	\onslide<3-> {\item Envoie des 'requêtes' aux autres classes}
	\newline
	\onslide<4-> {\item Interprète et agit en fonction des résultats}
	\newline
	\onslide<5-> {\item Difficulté : attendre le bon moment}
\end{itemize}
\end{frame}

	\subsection{Les joueurs}
\begin{frame}{Joueurs : virtualité et P.O.O}
\begin{itemize}
	\onslide<2-> {\item Classe virtuelle pure}
	\newline
	\onslide<3-> {\item Renvoie à 'Jeu' les informations}
	\newline
	\onslide<4-> {\item Sous-traite les choix}
\end{itemize}
\end{frame}

\begin{frame}{Les joueurs humains}
\begin{itemize}
	\onslide<2-> {\item Interaction entre l'homme et la machine}
	\newline
	\onslide<3-> {\item Interprète ce que l'humain décide}
	\newline
	\onslide<4-> {\item Sous traite le recueil de l'information aux 'Images'}
\end{itemize}
\end{frame}

\section{L'interface}
\begin{frame}{\textbf{L'interface} :Plan}
\begin{itemize}
	\onslide<2-> {\item Les Images}
	\newline
	\onslide<3-> {\item Boutons et Textes}
\end{itemize}
\end{frame}
	\subsection{Les Images}
\begin{frame}{La classe 'Images'}
\begin{itemize}
	\onslide<2-> {\item Intégration non parfaite aux classes : SDL $\Leftarrow$ C}
	\newline
	\onslide<3-> {\item Toutes cliquables}
	\newline
	\onslide<4-> {\item Peuvent toutes s'animer, avec ou sans transparence}
	\newline
	\onslide<5-> {\item Recueille l'information de l'humain et la renvoie à la machine}
\end{itemize}
\end{frame}
	\subsection{Boutons et Textes}
\begin{frame}{Interactions avec l'ordinateur}
	\onslide<2-> {
	\begin{block}{Boutons}
		\begin{itemize}
			\item Change de couleur lorsque la souris est dessus
			\item Classe d'agrément ...
		\end{itemize}
	\end{block}}
	\onslide<3-> {
	\begin{block}{Texte}
		\begin{itemize}
			\item Permet un rendu de texte sur l'écran
			\item Permet d'écrire (changement des noms)
			\item Attention aux fuites mémoires (changement de textes)
		\end{itemize}
	\end{block}}
\end{frame}	

\section{L'IA et le "machine learning"}
\begin{frame}{\textbf{L'IA et le "machine learning"} : Plan}
\begin{itemize}
	\onslide<2-> {\item Prendre et jouer}
	\newline
	\onslide<3-> {\item Le machine learning}
\end{itemize}
\end{frame}
	\subsection{Prendre et jouer}
\begin{frame}{l'Intelligence Artificielle}
	\onslide<2-> {
	\begin{block}{Prendre}
		\begin{itemize}
			\item \'Evaluation d'un score (nombre de points moyens espérés)
			\item Prise si score > 60
			\item Voir "machine learning"
		\end{itemize}
	\end{block}}
	\onslide<3-> {
	\begin{block}{Choisir une carte}
		\begin{itemize}
			\item Réaction parfois 'particulières'
			\item Défausse des As et des 10...
			\item Affectation du score en fonction de nombreux paramètres
			\item Choix de la carte avec le plus gros score
		\end{itemize}
	\end{block}}
\end{frame}
	\subsection{Le machine learning}
\begin{frame}{Modification des données chargées}
\begin{itemize}
	\onslide<2-> {\item Ne concerne que la partie "prise"}
	\newline
	\onslide<3-> {\item Modifie les données en fonction du résultat de la manche}
	\newline
	\onslide<4-> {\item Parties entre IA}
\end{itemize}
\end{frame}
\begin{frame}{Combats d'IA}
\begin{itemize}
	\onslide<2-> {\item Combat non implémentés, faute de temps}
	\newline
	\onslide<3-> {\item Plusieurs joueurs l'ont essayé $\Rightarrow$ plusieurs joueurs IA}
	\newline
	\onslide<4-> {\item Données rentrées}
	\newline
	\onslide<5-> {\item Attention à l'IA 'trop faible' }
	\onslide<5-> {\item Attention à l'IA 'trop faible' toto,tutu}
\end{itemize}
\end{frame}
\end{document}


